\documentclass{article}

\usepackage[english]{babel}
\usepackage[utf8]{inputenc}
\usepackage[T1]{fontenc}
\usepackage{listings}

\author{Florian Thuin \and Cyril de Vogelaere}
\title{Assignment 1: Getting familiar with the j-\phantom{}- compiler}
\date{\today}

\begin{document}
    \maketitle
    
    \section{Implementation}

	During this assignment, we modified the j-- grammar by adding three operators
	to the existing grammar.  
    
    \subsection{Division operator (/)}
    We followed step-by-step the first chapter of the book to implement the
    \textbf{division} operator. It helped us to understand where was each part
    of the compiler in the code and to have a look at what was already done and
    where we would have to add functionalities.

    \subsection{Modulo operator (\%)}
    The modulo operator is very similar to the division operator, such that
    each modification we made was nearly the same as what we did in the first
    part of this assignment. As it accepts the same limits (in terms of typing
    for example) as division, the idea behind the test was similar (only works
    on integer and sends back the rest of the division).

    \subsection{Substraction assignment operator (-=)}
    The substraction assignment operator was different from the two first
    operators because it is an \textbf{assignment} and not an operator. So we
    had to look at the \verb#JAssignment.java# file and verify what was already
    done for the \textit{addition assignment operator}. The main difference
    between those two operators is that the substraction doesn't work on
    \verb#String#. The tests made are verifying that it only works on \verb#int#
    and that it returns the expected result.
    
    \section{Testing}
    
    Before implementing each of our new operator, we wrote various tests that would 
    confirm whether our implementation worked or not. In the pass tests we tested various
    regular and edge cases to guarantee that the operator worked as intended provided that it
    received correct data. While in the fail tests incorrect cases such as the 
    division/modulo by 0 or other incorrect uses of the operators such as 
    effecting a subtraction assignment with an int in place of a variable. \newline
    
    The final build proved all those tests successful, allowing us 
    to guarantee the correctness of our final implementation.
    
    \section{Programming}
    
    Finally, we implemented the gcd function in j-\phantom{}-, using the previously described 
    substraction assignment operator :
    
    \lstinputlisting[language=Java, frame=single]{gcd.java}
    
    The tests were made as to ensure that it returned the expected result for
    limit cases (one parameter being 0, one parameter being 1, the two
    parameters not having a gcd greater than 1, the two parameters having
    a gcd greater than 1).

\end{document}
