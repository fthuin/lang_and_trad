\documentclass[a4paper, 11pt]{article}

\usepackage[english]{babel}
\usepackage[utf8]{inputenc}
\usepackage[T1]{fontenc}
\usepackage{placeins}
\usepackage{qtree}
\usepackage{csquotes}
\usepackage{hyperref}
\usepackage{amsmath}
\usepackage{tikz}
\usepackage{multicol}
\usetikzlibrary{automata,arrows,positioning,shapes}
\usepackage{graphicx}
\usepackage{amsmath}
\usepackage[left=2cm,right=2cm,top=2cm,bottom=2cm]{geometry}
\usepackage{listings}

\graphicspath{{img/}}

\usepackage{color}
% "define" Scala
\lstdefinelanguage{scala}{
  morekeywords={abstract,case,catch,class,def,%
    do,else,extends,false,final,finally,%
    for,if,implicit,import,match,mixin,%
    new,null,object,override,package,%
    private,protected,requires,return,sealed,%
    super,this,throw,trait,true,try,%
    type,val,var,while,with,yield},
  otherkeywords={=>,<-,<\%,<:,>:,\#},
  sensitive=true,
  morecomment=[l]{//},
  morecomment=[n]{/*}{*/},
  morestring=[b]",
  morestring=[b]',
  morestring=[b]"""
}

\definecolor{dkgreen}{rgb}{0,0.6,0}
\definecolor{gray}{rgb}{0.5,0.5,0.5}
\definecolor{mauve}{rgb}{0.58,0,0.82}

\lstset{frame=tb,
  language=scala,
  aboveskip=3mm,
  belowskip=3mm,
  showstringspaces=false,
  columns=flexible,
  basicstyle={\footnotesize\ttfamily},
  numbers=left,
  numberstyle=\tiny\color{gray},
  keywordstyle=\color{blue},
  commentstyle=\color{dkgreen},
  stringstyle=\color{mauve},
  frame=single,
  breaklines=true,
  breakatwhitespace=true,
  tabsize=3
}
\author{Florian Thuin \and Cyril de Vogelaere}
\date{\today}
\title{INGI2132 --- Domain Specific Language}

\begin{document}

    \maketitle
    %\tableofcontents

    \begin{abstract}
        In this report, we will show you how we build our DSL on top of the
        JavaMail library for the last assignment of the language and translators
        course.
    \end{abstract}

    \section{Functionalities and specificities of our DSL}

    We mainly focused on the sending of emails since we received an SMTP fake
    server and no POP3 or IMAP fake server, we thought it was the main part
    of the project. We implemented a lot of user-friendly shortcuts to add
    sender addresses, recipients, subject, texts,\ldots

    We hid a lot of the verbosity with a \verb#MessageInterface# trait that
    handles a lot of functions and also that contains the Message that reifies
    the email (this alone cuts off a big part of the code to write to send
    an email). \newline

    The user can easily apply his own configuration settings to send or receive
    emails from his own mail provider, thanks to this method:

    \lstinputlisting{mailSettings.scala}

    The user doesn't know that we set settings for both IMAP and SMTP, but he
    doesn't care since in one class he can only whether receive whether send
    emails.

    \subsection{Documented examples}
    % Some documented examples
    \subsubsection{Sending messages}
    As we get an example for the basic sending email approach, we tried to
    create a DSL on top of it. There is what it looks like at the end with
    every simple features given in the JavaMail library:

    \lstinputlisting{basicMessageSyntacticSugar.scala}

    \subsubsection{Receiving messages}

    We also tried to allow the user to easily receive email. It works great
    with Gmail.

    \lstinputlisting{receivingEmail.scala}

    \subsection{Advanced DSL features}
    % Where did you use advanced DSL features and why?
    We implemented a template-like loop such that the user can write an
    email to many people at the same time easily with customized information.

    \lstinputlisting{templateMessageDSL.scala}
    \subsection{Strong/weak points of our DSL}
    The strong point of our DSL is it is really easy to use, there is not really
    black magic behind it. It can be used to send a lot of emails that can
    be useful (not spams with everytime the same email sent for example). Our DSL
    doesn't reduce the scope of problems that can be solved by the JavaMail
    library and it should be easy to use our DSL along with other very specific
    usage. \newline

    The weak points of our DSL is that it is not very powerful to get the emails
    since the JavaMail API doesn't give much help (for example, in filtering
    emails with some kind of content). The other weak point resides in the fact
    that we use tuples for our template feature and tuples are defined by their
    length in Scala, it is not generalizable so we can't create a single method
    to handle each kind of template.

    \subsection{What if we had more time?}

    We would have made the template more flexible by accepting a bigger size of arguments
    and by reading from a CSV input for example. We could have made the DSL
    more flexible for reading a limited number of emails or for reading mails
    received between two dates. \newline

    We could also write more settings to allow the user to select YahooMail or
    Hotmail pre-defined settings. We also thought of having a folder of pre-written
    emails to send such that the user would have a method to send them all at
    the same time, it is an interesting feature but our friends found it
    counter-intuitive.

    \section{Conclusion}

    This was a very interesting project to work on. DSLs were a whole new
    field for us, it was sometimes hard to find ideas to implement and also
    hard to familiarize with Scala, but we learned a lot and it showed us how
    Scala can be powerful to handle common and not-so-common problems (from
    concatenating strings with commas to create customized loops) and we hope
    our project show that as a result.
\end{document}
