\documentclass[a4paper, 11pt]{article}

\usepackage[french]{babel}
\usepackage[utf8]{inputenc}
\usepackage[T1]{fontenc}
\usepackage{placeins}
\usepackage{qtree}
\usepackage{csquotes}
\usepackage{hyperref}
\usepackage{amsmath}
\usepackage{tikz}
\usepackage{multicol}
\usetikzlibrary{automata,arrows,positioning,shapes}
\usepackage{graphicx}
\usepackage{amsmath}
\usepackage[left=2cm,right=2cm,top=2cm,bottom=2cm]{geometry}
\usepackage{listings}

\graphicspath{{img/}}

\usepackage{color}
% "define" Scala
\lstdefinelanguage{scala}{
  morekeywords={abstract,case,catch,class,def,%
    do,else,extends,false,final,finally,%
    for,if,implicit,import,match,mixin,%
    new,null,object,override,package,%
    private,protected,requires,return,sealed,%
    super,this,throw,trait,true,try,%
    type,val,var,while,with,yield},
  otherkeywords={=>,<-,<\%,<:,>:,\#},
  sensitive=true,
  morecomment=[l]{//},
  morecomment=[n]{/*}{*/},
  morestring=[b]",
  morestring=[b]',
  morestring=[b]"""
}

\definecolor{dkgreen}{rgb}{0,0.6,0}
\definecolor{gray}{rgb}{0.5,0.5,0.5}
\definecolor{mauve}{rgb}{0.58,0,0.82}

\lstset{frame=tb,
  language=scala,
  aboveskip=3mm,
  belowskip=3mm,
  showstringspaces=false,
  columns=flexible,
  basicstyle={\footnotesize\ttfamily},
  numbers=left,
  numberstyle=\tiny\color{gray},
  keywordstyle=\color{blue},
  commentstyle=\color{dkgreen},
  stringstyle=\color{mauve},
  frame=single,
  breaklines=true,
  breakatwhitespace=true,
  tabsize=3
}
\author{Florian Thuin}
\date{\today}
\title{INGI2132 --- Domain Specific Language}

\begin{document}

    \maketitle
    \tableofcontents

    \section{Functionalities and specificities of our DSL}
    \subsection{Documented examples}
    % Some documented examples
    \subsubsection{Sending messages}
    As we get an example for the basic sending email approach, we tried to
    create a DSL on top of it. There is what it looks like at the end with
    every simple features given in the JavaMail library:

    \lstinputlisting{basicMessageSyntacticSugar.scala}

    We also tried to allow the user to easily receive email. It works great
    with Gmail.

    \lstinputlisting{receivingEmail.scala}

    \subsection{Advanced DSL features}
    % Where did you use advanced DSL features and why?
    We implemented a template-like loop such that the user can write an
    email to many people at the same time easily with customized information.

    \lstinputlisting{templateMessageDSL.scala}
    \subsection{Strong/weak points of our DSL}
    The strong point of our DSL is it is really easy to use, there is not a lot
    of black magic behind it. It can be used to send a lot of emails that can
    be useful (not spams with everytime the same email sent for example). \newline

    The weak points of our DSL is that it is not very powerful to get the emails
    since the JavaMail API doesn't give much help (for example, in filtering
    emails with some kind of content). The other weak point resides in the fact
    that we use tuples for our template feature and tuples are defined by their
    length in Scala, it is not generalizable so we can't create a single method
    to handle each kind of template.

    \subsection{What if we had more time?}

    We would have made the template more flexible by accepting a bigger size
    and by reading from a CSV input for example. We could have made the DSL
    more flexible for reading a limited number of emails or for reading mails
    received between two dates. \newline
\end{document}
