\documentclass[a4paper, 11pt]{article}

\usepackage[french]{babel}
\usepackage[utf8]{inputenc}
\usepackage[T1]{fontenc}
\usepackage{placeins}
\usepackage{csquotes}
\usepackage{hyperref}
\usepackage{amsmath}
\usepackage{tikz}
\usetikzlibrary{automata,arrows,positioning}

\author{Florian Thuin \and Cyril de Vogelaere}
\date{\today}
\title{Assignment 2: Deeper understanding of j-- compiler}

\begin{document}
    \maketitle
    \tableofcontents
    \section{Lexical analysis}
    Here is a regExp:
    $$ ((a^{*}b) \mid (ab)^{*})c $$

    With this regExp, we can define an infinite number of sequences:

    $$ \{ \epsilon, c, bc, abc, aabc, ababc, \ldots \} $$

    \subsection{NFA with Thompson construction}

    \begin{figure}[!ht]
    	\begin{center}
    		\begin{tikzpicture}[node distance=3cm, on grid, auto]
                \node[state, initial, initial text={}] (0) {$0$};
                \node[state] (1)  [above right =of 0] {$S_a$};
                \node        (Na) [right of = 1] {${N}_{a}$};
                \node[state] (2)  [right of = Na] {$E_{a}$};
                \node[state] (3)  [below right =of 0] {$3$};

                \path[->] (0) edge[bend left]   node [above] {$\epsilon$} (1);
                \path[->] (0) edge[bend right]  node [below] {$\epsilon$} (3);
            \end{tikzpicture}
    	\caption{State diagram}
    	\end{center}
    \end{figure}
    \section{Parsing}
    \section{DFA}
    \section{Language}
\end{document}
