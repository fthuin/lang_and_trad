\documentclass[a4paper, 11pt]{article}

\usepackage[french]{babel}
\usepackage[utf8]{inputenc}
\usepackage[T1]{fontenc}
\usepackage{placeins}
\usepackage{qtree}
\usepackage{csquotes}
\usepackage{hyperref}
\usepackage{amsmath}
\usepackage{tikz}
\usepackage{multicol}
\usetikzlibrary{automata,arrows,positioning,shapes}
\usepackage{graphicx}
\usepackage{amsmath}
\usepackage[left=2cm,right=2cm,top=2cm,bottom=2cm]{geometry}

\graphicspath{{img/}}

\author{Florian Thuin \and Cyril de Vogelaere}
\date{\today}
\title{Assignment 2: Deeper understanding of j-- compiler}

%NOTE, EN CAS DE BESOIN :
%First Set => https://www.youtube.com/watch?v=lqTwUxJ18d4
%Follow Set => https://www.youtube.com/watch?v=BuFhsJn3KPY
%Parsing Table => https://www.youtube.com/watch?v=HpmB3Wd8pxI

\begin{document}

    \maketitle
    \tableofcontents
    \clearpage{}

    \section{Recursive descent}
    \subsection{Changes to the grammar}

    	The current grammar is by no means $LL(1)$ as it is subject to both direct and
    	indirect recursion. The grammar must thus be changed as follow to become $LL(1)$:
    	\newline

    	% En cas de besoin : http://smlweb.cpsc.ucalgary.ca/start.html
    	\begin{flushleft}
    	E -> T $E_1$ \\
    	$E_1$ -> and T $E_1$ | nand T $E_1$ | $\epsilon$ \\
    	T -> F $T_1$ \\
		T1 -> or F $T_1$ | nor F $T_1$ | epsilon \\
		F -> ( E ) | !F | id \\
		id -> true | false \\
		\end{flushleft}

    	This new grammar created using the rules specified in the slides describes
    	the same language but allows a LL parser to be made.

\end{document}
