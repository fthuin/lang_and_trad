\documentclass[a4paper, 11pt]{article}

\usepackage[english]{babel}
\usepackage[utf8]{inputenc}
\usepackage[T1]{fontenc}
%\usepackage{placeins}
%\usepackage{qtree}
\usepackage{csquotes}
\usepackage{hyperref}
%\usepackage{amsmath}
%\usepackage{tikz}
%\usepackage{multicol}
%\usetikzlibrary{automata,arrows,positioning,shapes}
%\usepackage{graphicx}
\usepackage{amsmath}
%\usepackage[left=2cm,right=2cm,top=2cm,bottom=2cm]{geometry}

%\graphicspath{{img/}}

\author{Florian Thuin \and Cyril de Vogelaere}
\date{\today}
\title{Assignment 2: Deeper understanding of j-- compiler}

%NOTE, EN CAS DE BESOIN :
%First Set => https://www.youtube.com/watch?v=lqTwUxJ18d4
%Follow Set => https://www.youtube.com/watch?v=BuFhsJn3KPY
%Parsing Table => https://www.youtube.com/watch?v=HpmB3Wd8pxI

\begin{document}

    \maketitle
    %\tableofcontents

    \begin{abstract}
        Everything was submitted with the account of Cyril De Vogelaere.
        However, some submissions were made by Florian Thuin but you can
        ignore it. All the tests passed.
    \end{abstract}

    \section{Recursive descent}
    \subsection{Changes to the grammar}

    	The current grammar is by no means $LL(1)$ as it is subject to both direct and
    	indirect recursions. The grammar must thus be changed as follow to become $LL(1)$:
    	\newline

    	% En cas de besoin : http://smlweb.cpsc.ucalgary.ca/start.html
    	\begin{flushleft}
    	$E \rightarrow T\ E_1$ \\
    	$E_1 \rightarrow and\ T\ E_1\ \mid\ nand\ T\ E_1 \mid \epsilon$ \\
    	$T \rightarrow F\ T_1$ \\
		$T1 \rightarrow or\ F\ T_1 \mid nor\ F\ T_1 \mid \epsilon$ \\
		$F \rightarrow ( E ) \mid\ !F \mid id$ \\
		$id \rightarrow true \mid false$ \\
		\end{flushleft}

    	This new grammar created using the rules specified in the slides describes
    	the same language but allows a LL parser to be made. This was the grammar
    	implemented in our recursive descent parser.

    \section{Programming directly in Java bytecode}

    	Programming the bytecode was rather straightforward as we could easily compare
    	our produced output with a correctly compiled class file. Running tests was similarly
    	easy as we only had to compare the output of our file and the correct file, for a
    	given set of inputs.

    \section{Multi-line comments}

		Although the manual parsing part of the assignment was rather simple to
		implement in \verb#Scanner.java#, the javaCC parser part was a bit bewildering at first.
		\newline

		Once it was understood however --- thanks to the reading of the book --- it became
		rather easy as well. Our tests quickly revealed the few bugs that had escaped us,
		such as a multi-line comments that ends with the file.

	\section{The two for loops}

		This part of the assignment was actually quite a challenge, we had the
        bad idea to start with the \verb#j--.jj# part rather than the simpler
        manual part (because the same reasoning applies on both parts but
        the errors are more verbose with the Parser).\newline

		Another bad idea we had was to try to totally refactor for-each loop in
		basic for-loops. Which made sense at first as it would have allowed us
        to avoid the writing two \verb#codegen# and \verb#analyze# methods, but didn't in practice because we need to initialize the variable at the
        initialization of the loop and also after each iteration (we had a hard
        time with this part because the error message was talking about
        registers and it wasn't helpful). The remnant
        of that mistake can still be seen in our code but we now properly
        implement both methods and add a variable assignment at the correct
        position (in the initialization of the loop and before the body).\newline

		We eventually succeeded in both however by modifying \verb#Parser.java#,
        \verb#Scanner.java#, \verb#j--.jj#, and by adding some tokens
        (especially the modifiers) and some classes (\verb#JForStatement.java#
        and \verb#JForeachStatement.java# especially).\newline

		At first, the tests were failing as we were missing on too many cases on
        our tests (loop that did nothing, parse error, overflows,\ldots), we
        eventually figured it out however, as you can see in our final code
        which passes all the tests perfectly.

\end{document}
